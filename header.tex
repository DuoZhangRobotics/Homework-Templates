\usepackage{amsmath,amsfonts,amssymb,amsthm}
\newcommand\numberthis{\addtocounter{equation}{1}\tag{\theequation}}
\let\labelindent\relax
\usepackage{prettyref}
\usepackage{mathrsfs}
\usepackage{graphicx}
\usepackage{wrapfig}
\usepackage{subfig}
\usepackage{bbold}
\usepackage{tabu}
\usepackage{MnSymbol}
\usepackage{multirow}
\usepackage{booktabs}
\usepackage{enumitem}
\usepackage{algorithm}
\usepackage[table]{xcolor}
\usepackage[noend]{algpseudocode}
\usepackage
[backend=bibtex,
bibstyle=ieee,
citestyle=numeric,sortcites,
mincitenames=1,
maxcitenames=2,
natbib=true,
doi=false,
isbn=false,
url=false,
eprint=false]{biblatex}
\usepackage[colorlinks,allcolors=gray,hypertexnames=true]{hyperref} 
\newcommand{\citewithauthor}[1]{\citeauthor{#1} \cite{#1}}

\newtheorem{theorem}{\TE{Theorem}}[section]
\newtheorem{lemma}[theorem]{\TE{Lemma}}
\newtheorem{remark}[theorem]{\TE{Remark}}
\newtheorem{assume}[theorem]{\TE{Assumption}}
\newtheorem{proposition}[theorem]{\TE{Proposition}}
\newtheorem{corollary}[theorem]{\TE{Corollary}}
\newtheorem{define}[theorem]{\TE{Definition}}
\algnewcommand{\IfThenElse}[3]{% \IfThenElse{<if>}{<then>}{<else>}
  \State \algorithmicif\ #1\ \algorithmicthen\ #2\ \algorithmicelse\ #3}
  
\algnewcommand{\LineComment}[1]{\State \(\triangleright\) #1}
\algdef{SE}[DOWHILE]{Do}{doWhile}{\algorithmicdo}[1]{\algorithmicwhile\ #1}
% Syntax: \colorboxed[<color model>]{<color specification>}{<math formula>}
\newcommand*{\colorboxed}{}
\def\colorboxed#1#{%
  \colorboxedAux{#1}%
}
\newcommand*{\colorboxedAux}[3]{%
  % #1: optional argument for color model
  % #2: color specification
  % #3: formula
  \begingroup
    \colorlet{cb@saved}{.}%
    \color#1{#2}%
    \boxed{%
      \color{cb@saved}%
      #3%
    }%
  \endgroup
}
% ------------------------------------------------------------------------

% ------------------------------------------------------------------------equation
\newrefformat{fig}{Figure~\ref{#1}}
\newrefformat{par}{Section~\ref{#1}}
\newrefformat{appen}{Appendix~\ref{#1}}
\newrefformat{sec}{Section~\ref{#1}}
\newrefformat{sub}{Section~\ref{#1}}
\newrefformat{table}{Table~\ref{#1}}
\newrefformat{ass}{Assumption~\ref{#1}}
\newrefformat{alg}{Algorithm~\ref{#1}}
\newrefformat{def}{Definition~\ref{#1}}
\newrefformat{thm}{Theorem~\ref{#1}}
\newrefformat{cor}{Corollary~\ref{#1}}
\newrefformat{lem}{Lemma~\ref{#1}}
\newrefformat{step}{Step~\ref{#1}}
\newrefformat{ln}{Line~\ref{#1}}
\newrefformat{rem}{Remark~\ref{#1}}
\newrefformat{eq}{Equation~\eqref{#1}}
\newrefformat{pb}{Problem~\ref{#1}}
\newrefformat{it}{Item~\ref{#1}}
\newrefformat{te}{Term~\ref{#1}}
\def\Eqref Eq:#1:{\eqref{eq:#1}}
\newrefformat{Eq}{Equation~\Eqref#1:}
\newcommand{\prettyreft}[1]{\text{\prettyref{#1}}}
\renewcommand{\algorithmicrequire}{\textbf{Input:}}
\renewcommand{\algorithmicensure}{\textbf{Output:}}
\newcommand{\shortref}[2]{Fig.\ref{#1}#2}
\newcommand{\CTR}{\textbf{:}}
\newcommand{\IDD}{\operatorname{Id}}
\newcommand{\COMP}{\mathcal{O}}
\newcommand{\T}[1]{\tilde{#1}}
\newcommand{\R}{\mathbb{R}}
\newcommand{\E}[1]{\mathbf{#1}}
\newcommand{\TE}[1]{\textbf{#1}}
\newcommand{\B}[1]{\boldsymbol{#1}}
%\newcommand{\FPPAT}[3]{\frac{\partial{#1}}{\partial{#2}}\Bigr|_{#3}}
\newcommand{\ABS}[1]{\left| #1 \right|}
\newcommand{\NORM}[1]{\left\lVert #1 \right\rVert}
\newcommand{\FPPAT}[3]{\frac{\partial{#1}}{\partial{#3}}}
\newcommand{\FPP}[2]{\frac{\partial{#1}}{\partial{#2}}}
\newcommand{\FPPR}[2]{{\partial{#1}}/{\partial{#2}}}
\newcommand{\FPPT}[2]{\frac{\partial^2{#1}}{\partial{#2}^2}}
\newcommand{\FPPTTT}[2]{\frac{\partial^3{#1}}{\partial{#2}^3}}
\newcommand{\FPPTT}[3]{\frac{\partial^2{#1}}{\partial{#2}\partial{#3}}}
\newcommand{\FPPM}[3]{\frac{\partial^2{#1}}{\partial{#2}\partial{#3}}}
\newcommand{\FDD}[2]{\frac{d{#1}}{d{#2}}}
\newcommand{\FDM}[2]{\frac{D{#1}}{D{#2}}}
\newcommand{\TWO}[2]{\left(\setlength{\arraycolsep}{1pt}\begin{array}{cc}{#1}, & {#2}\end{array}\right)}
\newcommand{\TWOC}[2]{\left(\setlength{\arraycolsep}{1pt}\begin{array}{c}#1 \\ #2\end{array}\right)}
\newcommand{\TWOR}[2]{\left(\setlength{\arraycolsep}{1pt}\begin{array}{cc}{#1}^T, & {#2}^T\end{array}\right)^T}
\newcommand{\THREE}[3]{\left(\setlength{\arraycolsep}{1pt}\begin{array}{ccc}{#1}, & {#2}, & {#3}\end{array}\right)}
\newcommand{\THREENB}[3]{\setlength{\arraycolsep}{1pt}\begin{array}{ccc}{#1}, & {#2}, & {#3}\end{array}}
\newcommand{\THREEC}[3]{\left(\setlength{\arraycolsep}{1pt}\begin{array}{c}#1 \\ #2 \\ #3\end{array}\right)}
\newcommand{\THREER}[3]{\left(\setlength{\arraycolsep}{1pt}\begin{array}{ccc}{#1}^T, & {#2}^T, & {#3}^T\end{array}\right)^T}
\newcommand{\FOUR}[4]{\left(\setlength{\arraycolsep}{1pt}\begin{array}{cccc}{#1}, & {#2}, & {#3}, & {#4}\end{array}\right)}
\newcommand{\FOURC}[4]{\left(\setlength{\arraycolsep}{1pt}\begin{array}{c}#1 \\ #2 \\ #3 \\ #4\end{array}\right)}
\newcommand{\FOURR}[4]{\left(\setlength{\arraycolsep}{1pt}\begin{array}{cccc}{#1}^T, & {#2}^T, & {#3}^T, & {#4}^T\end{array}\right)^T}
\newcommand{\FIVE}[5]{\left(\setlength{\arraycolsep}{1pt}\begin{array}{ccccc}{#1}, & {#2}, & {#3}, & {#4}, & {#5}\end{array}\right)}
\newcommand{\FIVEC}[5]{\left(\setlength{\arraycolsep}{1pt}\begin{array}{c}#1 \\ #2 \\ #3 \\ #4 \\ #5\end{array}\right)}
\newcommand{\FIVER}[5]{\left(\setlength{\arraycolsep}{1pt}\begin{array}{ccccc}{#1}^T, & {#2}^T, & {#3}^T, & {#4}^T, & {#5}^T\end{array}\right)^T}
\newcommand{\SIX}[6]{\left(\setlength{\arraycolsep}{1pt}\begin{array}{cccccc}{#1}, & {#2}, & {#3}, & {#4}, & {#5}, & {#6}\end{array}\right)}
\newcommand{\SIXC}[6]{\left(\setlength{\arraycolsep}{1pt}\begin{array}{c}#1 \\ #2 \\ #3 \\ #4 \\ #5 \\ #6\end{array}\right)}
\newcommand{\SIXR}[6]{\left(\setlength{\arraycolsep}{1pt}\begin{array}{cccccc}{#1}^T, & {#2}^T, & {#3}^T, & {#4}^T, & {#5}^T, & {#6}^T\end{array}\right)^T}
\newcommand{\MTT}[4]{\left(\setlength{\arraycolsep}{1pt}\begin{array}{cc}#1 & #2 \\ #3 & #4\end{array}\right)}
\newcommand{\MTTF}[9]{\left(\setlength{\arraycolsep}{1pt}
\begin{array}{ccccc}
#1 & #2 & #3 & #4 & #5\\
#6 & & & &\\
#7 & & & &\\
#8 & & & &\\
#9 & & & &
\end{array}\right)}
\newcommand{\MTTT}[9]{\left(\setlength{\arraycolsep}{1pt}\begin{array}{ccc}#1 & #2 & #3 \\ #4 & #5 & #6 \\ #7 & #8 & #9\end{array}\right)}
\newcommand{\MDD}[3]{\left(\setlength{\arraycolsep}{1pt}\begin{array}{ccc}#1 & & \\ & #2 & \\ & & #3\end{array}\right)}
\newcommand{\tr}[1]{\text{tr}(#1)}
\newcommand{\dist}{\text{dist}}
\newcommand{\adj}{\mathcal{M}}
\newcommand{\PRIME}{^\prime}
\newcommand{\BRACKETS}[1]{\left( #1 \right)}
\newcommand{\downup}[2]{_{#1}^{#2}}
\newcommand{\LIST}[1]{\left\{ #1 \right.}
\newenvironment{eqntiny}{\begin{tiny}\begin{eqnarray*}}{\end{eqnarray*}\end{tiny}}
\newenvironment{eqntinyN}{\begin{tiny}\begin{eqnarray}}{\end{eqnarray}\end{tiny}}
\newcommand{\fmin}[1]{\underset{#1}{\min}}
\newcommand{\fmax}[1]{\underset{#1}{\max}}
\newcommand{\argcon}{\text{s.t.}\quad}
\newcommand{\argmin}[1]{\underset{#1}{\text{argmin}}\;}
\newcommand{\argmax}[1]{\underset{#1}{\text{argmax}}\;}
\newcommand{\ST}{\text{s.t.}\quad}
\newcommand{\verteq}{\rotatebox{-90}{$\triangleq$}}
\newcommand{\definedAs}[2]{\underset{\scriptstyle\overset{\mkern4mu\verteq}{#2}}{\underline{#1}}}
%\newcommand{\C}[1]{\left[{#1}\right]_\times}
\newcommand{\TWORCell}[2]{\begin{tabular}{@{}c@{}}#1 \\ #2\end{tabular}}
\newcommand{\THREERCell}[3]{\begin{tabular}{@{}l@{}}#1 \\ #2 \\ #3\end{tabular}}
% ------------------------------------------------------------------------

% ------------------------------------------------------------------------comments
\usepackage{xcolor}
\definecolor{Blue} {rgb}{0.4, 0.4, 0.8}
\definecolor{Red}  {rgb}{0.8, 0.2, 0.2}
\definecolor{Green}{rgb}{0.2, 0.8, 0.2}
\newcommand{\revised}[1]{\textcolor{Blue}{#1}}
\newenvironment{revisedBlk}{\textcolor{Blue}}{\textcolor{black}}
\newcommand{\kui}[1]{\textcolor[RGB]{76,153,0}{[\textbf{KUI:} {\em #1}]}}
\newcommand{\xifeng}[1]{\textcolor[RGB]{0,76,150}{[\textbf{XG:} {\em #1}]}}
\newcommand{\zherong}[1]{\textcolor[RGB]{153,76,0}{[\textbf{ZP:} {\em #1}]}}
% ------------------------------------------------------------------------